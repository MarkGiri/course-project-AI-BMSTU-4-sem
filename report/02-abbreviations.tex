\phantomsection
\part*{ОПРЕДЕЛЕНИЯ, ОБОЗНАЧЕНИЯ И\\СОКРАЩЕНИЯ}
\addcontentsline{toc}{chapter}{\textbf{ОПРЕДЕЛЕНИЯ, ОБОЗНАЧЕНИЯ И СОКРАЩЕНИЯ}}
В настоящей расчетно-пояснительной записке применяются следующие термины с соответствующими определениями.

\begin{enumdescript}
	\item[Сложный вопрос] -- вопрос, содержащий нескольк аспектов и требующий многокомпонентного ответа.
	
	\item[Простой вопрос] -- самодостаточный атомарный вопрос, фокусирующийся на одном аспекте с определенным ответом.
	
	\item[Модель] -- упрощённое представление реального объекта, процесса или явления, отражающее его существенные свойства.
	
	\item[Языковая модель] -- алгоритмическая система, обученная на больших массивах текста для понимания и генерации естественного языка.
	
	\item[Декомпозиция] -- разделение целого на составные части.
	
	\item[Промпт] -- текстовая инструкция для языковой модели.
	
	\item[Токен] -- минимальная единица текста, обрабатываемая языковой моделью.
	
	\item[ИИ] -- искусственный интеллект.
	
	\item[LLM] -- Large Language Model (большая языковая модель).
	
	\item[QDMR] -- Question Decomposition Meaning Representation (представление смысла декомпозиции вопроса).
	
	\item[EDG] -- Entity Description Graph (граф описания сущностей).
	
	\item[HSP] -- Hierarchical Semantic Parsing (иерархический семантический парсинг).
	
	\item[IDEF0] -- Integration DEFinition for Function Modeling (методология функционального моделирования).
	
	\item[Dependency-граф] -- граф, отражающий синтаксические связи между словами в предложении.
	
	\item[API] -- Application Programming Interface (программный интерфейс приложения).
	
	\item[MMLU] -- Massive Multitask Language Understanding (массовое многозадачное понимание языка).
	
	\item[MATH] -- Mathematics Dataset (набор математических задач для тестирования языковых моделей).
	
	\item[HumanEval] -- набор тестов для оценки способности модели генерировать код.
	
	\item[HellaSwag] -- набор тестов для оценки здравого смысла и понимания контекста.
	
	\item[GPQA] -- General Purpose Question Answering (тест для оценки способности отвечать на общие вопросы).
	
	\item[Квантование] -- метод оптимизации модели путем уменьшения точности представления весов.
	
	\item[Дистилляция] -- процесс передачи знаний от большой модели к меньшей.
	
	\item[Токенизация] -- процесс разбиения текста на минимальные единицы (токены).
	
	\item[Контекстное окно] -- максимальное количество токенов, которое модель может обработать за один раз.
\end{enumdescript}