\chapter{Организационно-правовой раздел}

\section{Лицензионная политика проекта}

Программное решение распространяется под лицензией Apache 2.0 \cite{apache}, что обусловлено требованиями используемой языковой модели T-lite-it-1.0 \cite{tlite} от T Bank и необходимостью обеспечения совместимости с компонентами, лицензированными под MIT \cite{mit}.

\subsection{Обоснование выбора лицензии}

Проект включает библиотеки \texttt{typer} \cite{typer}, \texttt{pydantic} \cite{pydantic}, \texttt{rich} \cite{rich} и \texttt{llama-cpp-python} \cite{llama-cpp-python}, распространяемые под лицензией MIT. Совместимость лицензий MIT и Apache 2.0 подтверждается сравнительным анализом их требований (табл.~\ref{tab:license-comparison}). Текст лицензии Apache 2.0 находится в приложении A.

\begin{table}[ht]
	\centering
	\caption{Сравнительный анализ лицензий MIT и Apache 2.0}
	\label{tab:license-comparison}
	\begin{tabular}{|p{4.5cm}|p{3.5cm}|p{3.5cm}|p{2.5cm}|}
		\hline
		\textbf{Критерий} & \textbf{Лицензия MIT} & \textbf{Лицензия Apache 2.0} & \textbf{Сравнение} \\
		\hline
		Коммерческое использование & Разрешено & Разрешено & Одинаково \\
		\hline
		Модификация & Разрешено без ограничений & Разрешено с требованием указания изменений & Apache 2.0 строже \\
		\hline
		Распространение & Разрешено & Разрешено с дополнительными требованиями & Apache 2.0 строже \\
		\hline
		Частное использование & Разрешено & Разрешено & Одинаково \\
		\hline
		Указание изменений & Не требуется явно & Требуется явное указание & Apache 2.0 строже \\
		\hline
		Защита авторских прав & Требует включения уведомления & Требует сохранения всех уведомлений & Apache 2.0 строже \\
		\hline
		Патентные права & Нет явного предоставления & Явное предоставление патентных прав & Apache 2.0 строже \\
		\hline
		Ограничение торговых марок & Отсутствует & Прямо запрещает использование & Apache 2.0 строже \\
		\hline
		Распространение производных & Разрешено без особых условий & Разрешено с дополнительными условиями & Apache 2.0 строже \\
		\hline
	\end{tabular}
\end{table}

\subsection{Правовые аспекты использования лицензии}

Анализ таблицы~\ref{tab:license-comparison} показывает, что Apache 2.0 по всем критериям либо эквивалентна MIT, либо накладывает более строгие ограничения. Согласно принципам лицензионной совместимости, подтвержденным \mbox{Open Source Initiative \cite{osi}}, компоненты с более открытой лицензией (MIT) могут включаться в проекты под более строгой лицензией (Apache 2.0) при соблюдении требований обеих лицензий.

Ключевые преимущества выбранной лицензионной политики:

\begin{itemize}
	\item правовая защита посредством явного предоставления патентных прав;
	\item чёткие требования к указанию модификаций, обеспечивающие прозрачность изменений;
	\item сохранение авторских прав разработчиков;
	\item возможность коммерческого использования с соблюдением условий лицензии.
\end{itemize}

Для учебных целей применение программного решения дополнительно защищено ст. 1274 ГК РФ, разрешающей свободное использование объектов авторского права в образовательном процессе.

\section{Меры правовой безопасности}

\subsection{Многоуровневая валидация контента}
Фильтрация контента обязательна согласно ст. 10.1 ФЗ-149 <<Об информации, информационных технологиях и о защите информации>>, требующей блокировки запрещённой информации (экстремизм, призывы к насилию), а также ФЗ-114 <<О противодействии экстремистской деятельности>>. Несоблюдение влечёт ответственность по ст. 282 УК РФ и ст. 13.41 КоАП РФ, а также риск блокировки проекта.

Архитектура системы включает комплексный подход к обеспечению правовой безопасности, сочетающий жёсткую и мягкую фильтрацию контента:

\begin{enumerate}
	\item \textbf{жёсткая фильтрация} основана на автоматической проверке текста по спискам запрещённых слов и словосочетаний, формируемым на основе официальных источников:
	\begin{itemize}
		\item список экстремистских материалов Министерства юстиции РФ \cite{minust};
		\item единый реестр запрещенной информации Роскомнадзора \cite{rkn};
		\item перечень запрещённой информации согласно постановлениям Верховного Суда РФ \cite{verh_sud};
		\item список экстремистских и террористических организаций, признанных таковыми Генеральной прокуратурой РФ \cite{prokur};
		\item рекомендации Минцифры \cite{min_cifr}.
	\end{itemize}
	
	\item \textbf{мягкая фильтрация} реализуется посредством нейросетевой модели, которая:
	\begin{itemize}
		\item на этапе предварительной обработки получает системные инструкции с явными запретами на генерацию контента, нарушающего законодательство РФ, включая ст. 282 УК РФ, ФЗ-149;
		\item выполняет семантический анализ содержания, способный идентифицировать противоправный контент даже при использовании эвфемизмов или нестандартных формулировок.
	\end{itemize}
\end{enumerate}

На этапе постобработки ответ модели анализируется через дополнительный валидационный запрос. При обнаружении потенциальных нарушений система блокирует выдачу контента и прекращает обработку запроса.

\subsection{Формирование и обновление списков запрещённых слов}
Базовые списки запрещённых слов и выражений формируются на основе официальных источников регулирующих органов РФ, которые были указаны выше.

\textbf{Обязанность пользователя:} В соответствии с требованиями ст. 10.1 \mbox{ФЗ-149}, пользователь несёт полную ответственность за:

\begin{itemize}
	\item регулярное обновление встроенных списков запрещённых слов и выражений из вышеуказанных официальных источников, в соответствии с изменениями в действующем законодательстве;
	\item самостоятельное дополнение базовых списков с учётом специфики использования программы и отраслевых требований;
	\item проверку актуальности используемых фильтров перед каждым применением программного решения.
\end{itemize}

Система предоставляет техническую возможность для обновления списков, но \textbf{не осуществляет их автоматическое обновление}. Вся ответственность за актуальность и полноту списков запрещённых слов и выражений лежит исключительно на пользователе программного решения.

\subsection{Обработка персональных данных}
В соответствии с требованиями ФЗ-152 <<О персональных данных>> и \mbox{ФЗ-149} <<Об информации, информационных технологиях и о защите информации>>:

\begin{itemize}
	\item программное решение \textbf{не предназначено} для обработки персональных данных и \textbf{не должно использоваться} для этих целей;
	\item система настроена на блокировку генерации и обработки контента, содержащего персональные данные;
	\item \textbf{пользователю категорически запрещается} вводить персональные данные субъектов в систему;
	\item в случае ввода персональных данных в систему вся ответственность за их обработку в соответствии со ст. 13.11 КоАП РФ и ФЗ-152 возлагается исключительно на пользователя.
\end{itemize}

При обнаружении в запросе признаков персональных данных система предупреждает пользователя о недопустимости их обработки и может отказать в выполнении запроса.

\subsection{Ответственность пользователя}
Перед каждым использованием программы пользователь обязан:
\begin{itemize}
	\item самостоятельно убедиться в актуальности используемых списков запрещённых слов и выражений путём проверки официальных источников;
	\item обновить базы данных запрещённых выражений в соответствии с изменениями в законодательстве;
	\item дополнить систему фильтрации собственными ограничениями в соответствии с конкретными задачами использования системы;
	\item верифицировать результаты работы системы на соответствие действующему законодательству РФ.
\end{itemize}

\newpage
В соответствии с ст. 13.41 КоАП РФ и общими положениями гражданского законодательства, пользователь несёт полную юридическую ответственность за:
\begin{itemize}
	\item актуальность и полноту используемых фильтров запрещённой информации;
	\item недопущение обработки персональных данных с использованием системы;
	\item использование и распространение информации, полученной с помощью данного программного обеспечения.
\end{itemize}

Разработчик не несёт ответственности за последствия использования программы в нарушение действующего законодательства, несмотря на наличие встроенных механизмов защиты.

\subsection{Ограничения и риски}
Система подвержена следующим технико-правовым ограничениям: появление ложных срабатываний при обработке специализированных научных, медицинских и юридических терминов; возможность обхода фильтров через сложные лингвистические конструкции; недостаточная точность интерпретации контекста при анализе многозначных выражений. Эффективность фильтрации зависит от регулярности обновления пользователем баз запрещённых выражений в соответствии с изменениями законодательства.

\section{Вывод}
Реализованные правовые механизмы обеспечивают соответствие проекта требованиям ФЗ-149, ФЗ-114 и УК РФ за счёт многоуровневой фильтрации контента, основанной на официальных списках запрещённых материалов и нейросетевом анализе. Лицензия Apache 2.0 гарантирует совместимость с компонентами MIT, а распределение ответственности (обновление фильтров, запрет на обработку персональных данных) минимизирует юридические риски.