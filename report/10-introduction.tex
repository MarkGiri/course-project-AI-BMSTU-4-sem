\phantomsection
\part*{ВВЕДЕНИЕ}
\addcontentsline{toc}{chapter}{\textbf{ВВЕДЕНИЕ}}

Развитие систем искусственного интеллекта создаёт новые возможности для автоматизации сложных интеллектуальных задач. Одной из таких задач является автоматическое разбиение сложных вопросов на простые составляющие. Это особенно актуально в контексте развития больших языковых моделей, которые показывают высокую эффективность при работе с простыми запросами, но часто затрудняются при обработке сложных (комплексных) вопросов. \cite{press2023measuring}

Существующие подходы к декомпозиции вопросов включают как классические методы на основе лингвистического анализа, так и современные решения с использованием нейронных сетей и больших языковых моделей. При этом классические методы часто ограничены жёсткими правилами и шаблонами, в то время как методы на основе ИИ способны адаптироваться к различным формулировкам и контекстам.

Цель работы состоит в разработке программного решения для автоматического разбиения сложных вопросов на простые с использованием больших языковых моделей.

Для достижения поставленной цели определены следующие задачи:

\begin{enumdescript}
	\item анализ существующих методов декомпозиции вопросов;
	\item разработка алгоритма декомпозиции на основе больших языковых моделей;
	\item реализация программного решения;
	\item исследование эффективности разработанного решения.
\end{enumdescript}