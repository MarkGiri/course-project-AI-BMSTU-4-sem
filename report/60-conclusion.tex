\phantomsection
\part*{ЗАКЛЮЧЕНИЕ}
\addcontentsline{toc}{chapter}{\textbf{ЗАКЛЮЧЕНИЕ}}

В рамках курсового проекта разработано программное решение для автоматической декомпозиции сложных вопросов на простые с использованием больших языковых моделей. 

Анализ существующих подходов показал преимущества применения языковых моделей над экспертными и алгоритмическими методами: отсутствие необходимости в разработке сложных правил, высокая адаптивность к различным типам вопросов и эффективное сохранение контекста между частями исходного вопроса.

Разработанная архитектура включает три взаимосвязанных модуля: консольный интерфейс, управление запросами и взаимодействие с языковой моделью. Реализация на Python обеспечивает работу на стандартном оборудовании без необходимости в графических ускорителях. Внедрена двухуровневая система валидации контента для входных запросов и генерируемых ответов.

Экспериментальное исследование выявило, что проприетарные модели (DeepSeek-V3-Chat, ChatGPT-4, Claude-3-Opus) демонстрируют наивысшее качество декомпозиции, однако требуют постоянного подключения к интернету. Среди локальных решений выделяются T-Tech-T-pro-it-1.0 и RuadaptQwen-32B-Pro\_v1. Для практической реализации выбрана компактная квантованная модель T-lite-it-1.0-q4\_k\_m как оптимальный компромисс между качеством и доступностью на пользовательском оборудовании.

Программное решение распространяется под лицензией Apache 2.0 и соответствует требованиям законодательства РФ благодаря реализованной фильтрации контента на основе официальных источников.

В результате созданное программное решение успешно выполняет декомпозицию сложных вопросов на простые компоненты и может применяться в образовательных системах, системах поддержки принятия решений и других приложениях обработки естественного языка.