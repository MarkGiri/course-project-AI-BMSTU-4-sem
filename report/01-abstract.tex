\phantomsection
\part*{РЕФЕРАТ}
\addcontentsline{toc}{chapter}{\textbf{РЕФЕРАТ}}

Расчетно-пояснительная записка \pageref{LastPage} с., \totalfigures\ рис., \totaltables\ табл., \total{citnum} ист., \total{appendix}\ прил.

В работе представлено программное решение на основе искусственного интеллекта для разбиения сложных вопросов на простые с использованием больших языковых моделей.

Ключевые слова: большие языковые модели, декомпозиция вопросов, искусственный интеллект, обработка естественного языка, промпт-инженерия.

Проведен комплексный анализ существующих методов декомпозиции вопросов, включая экспертную оценку, алгоритмические методы и методы на основе больших языковых моделей. Разработана трёхуровневая архитектура программного решения, состоящая из модуля консольного интерфейса, модуля управления запросами и модуля взаимодействия с языковой моделью. Реализованы алгоритмы формирования запроса к языковой модели и обработки ответа с многоуровневой валидацией контента. Выполнено исследование эффективности декомпозиции на тестовой выборке из 10 сложных вопросов с экспертной оценкой полноты, атомарности и корректности результатов для 30 языковых моделей. Установлено, что наилучшими показателями обладают проприетарные модели (DeepSeek-V3-Chat, ChatGPT-4, Claude-3-Opus), а среди локальных решений высокую эффективность демонстрируют T-Tech-T-pro-it-1.0 и RuadaptQwen-32B-Pro\_v1. Разработаны правовые механизмы защиты системы в соответствии с требованиями законодательства РФ.