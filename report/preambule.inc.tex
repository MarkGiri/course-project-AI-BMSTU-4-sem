\usepackage{polyglossia}
\setmainlanguage[babelshorthands = true]{russian}
\setotherlanguage{english}

%\usepackage{newtxmath}
\usepackage[no-math]{fontspec}

\defaultfontfeatures{Ligatures = TeX, Mapping = tex-text}

\setmainlanguage[babelshorthands = true]{russian}
\setotherlanguage{english}

\setmainfont{Times New Roman}

\newfontfamily\cyrillicfont{Times New Roman}
\newfontfamily\englishfont{Times New Roman}

\newfontfamily\cyrillicfonttt{Courier New}[Script=Cyrillic]
\newfontfamily\englishfonttt{Courier New}

\usepackage[
left=30mm,
right=10mm, 
top=20mm,
bottom=20mm,
]{geometry}

\makeatletter
	\renewcommand\LARGE{\@setfontsize\LARGE{22pt}{20}}
	\renewcommand\Large{\@setfontsize\Large{20pt}{20}}
	\renewcommand\large{\@setfontsize\large{16pt}{20}}
\makeatother

\usepackage{microtype} % Настройка переносов
\sloppy

\usepackage{setspace} % Настройка межстрочного интервала
\onehalfspacing

\usepackage{indentfirst} % Настройка абзацного отступа
\setlength{\parindent}{12.5mm}

\usepackage[unicode,hidelinks]{hyperref}
\usepackage{xifthen}

\usepackage{colortbl}

\usepackage[normalem]{ulem}
% Текст под линией 
\newcommand*{\undertext}[2]{%
	\begin{tabular}[t]{@{}c@{}}%
		#1\\\relax\scriptsize(#2)%
	\end{tabular}
}

% горизонтальная линия
\makeatletter
\newcommand{\vhrulefill}[1]
{
	\leavevmode\leaders\hrule\@height#1\hfill \kern\z@
}

\usepackage[figure,table]{totalcount} % Подч изображений, таблиц
\usepackage{rotating} % Поворот изображения вместе с названием
\usepackage{lastpage} % Для подсчета числа страниц

\usepackage{titlesec}
\usepackage{titletoc}
\usepackage{tocloft}
\usepackage{pdfpages}

\setcounter{tocdepth}{5}

\setlength{\cftbeforetoctitleskip}{-25pt}
\renewcommand{\cfttoctitlefont}{\large\bfseries}

\renewcommand{\cftchapfont}{\large\bfseries}
\renewcommand{\cftsecfont}{\large}
\renewcommand{\cftchapleader}{\cftdotfill{\cftdotsep}}
\renewcommand{\cftpartleader}{\cftdotfill{\cftdotsep}}

\setlength{\cftbeforepartskip}{10pt}


\makeatletter
\renewcommand{\toclevel@part}{0}
\renewcommand{\toclevel@chapter}{0}
\renewcommand{\toclevel@section}{1}
\renewcommand{\toclevel@subsection}{2}
\makeatother
\usepackage{bookmark}

\setcounter{secnumdepth}{5}

\titleformat{\part}[block]
{\large\bfseries}{\hspace{12.5mm}\thechapter}{0.5em}{\large\centering}

\titleformat{\chapter}[block]
{\large\bfseries}{\hspace{12.5mm}\thechapter}{0.5em}{\large\raggedright}

\titleformat{\section}[block]
{\large\bfseries}{\hspace{12.5mm}\thesection}{0.5em}{\large\raggedright}
\renewcommand{\thesection}{\arabic{chapter}.\arabic{section}} 

\titleformat{\subsection}[block]
{\large\bfseries}{\hspace{12.5mm}\thesubsection}{0.5em}{\large\raggedright}
\renewcommand{\thesubsection}{\arabic{chapter}.\arabic{section}.\arabic{subsection}}

\titleformat{\subsubsection}[block]
{\large\bfseries}{\hspace{12.5mm}\thesubsubsection}{0.5em}{\large\raggedright}
\renewcommand{\thesubsubsection}{\arabic{chapter}.\arabic{section}.\arabic{subsection}.\arabic{subsubsection}}

\titleclass{\part}{top}
\titlespacing*{\part}{12.5mm}{-22pt}{10pt}

\titlespacing{\chapter}{0pt}{-22pt}{10pt}
\titlespacing{\section}{0pt}{10pt}{10pt}
\titlespacing{\subsection}{0pt}{10pt}{10pt}
\titlespacing{\subsubsection}{0pt}{10pt}{10pt}

% ---------------------------------------- CAPTION --------------------------------- %

\usepackage[
	labelsep=endash,
	singlelinecheck=false,
]{caption}

\captionsetup[figure]{justification=centering, name=Рисунок}
\captionsetup[table]{justification=raggedleft}
\captionsetup[listing]{justification=raggedright}


% ---------------------------------------- ABBRS --------------------------------- %

\usepackage{enumitem}
\newcounter{descriptcount}
\newlist{enumdescript}{description}{2}
\setlist[enumdescript,1]{
	leftmargin=\parindent,
	itemindent=5pt,
	labelsep=0.25em,
	align=left,
	parsep=0pt,
	itemsep=0pt,
	topsep=0pt,
	before={\setcounter{descriptcount}{0}
		\renewcommand*\thedescriptcount{\arabic{descriptcount})}
		\setlength{\parindent}{12.5mm}},
	font=\normalfont\stepcounter{descriptcount}\thedescriptcount~
}
\setlist[enumdescript,2]{
	before={\setcounter{descriptcount}{0}
		\renewcommand*\thedescriptcount{\roman{descriptcount}.}}
	,font=\normalfont\stepcounter{descriptcount}\thedescriptcount~
}

\def\labelitemi{---} % Изменение буллета для списков
\setlist[itemize]{
	leftmargin=\parindent,
	itemindent=0pt,
	labelwidth=15pt,
	labelsep=2.5mm,
	labelindent=\dimexpr\parindent-10mm-2.5mm\relax,
	align=left,
	parsep=0pt,
	itemsep=0pt,
	topsep=0pt,
	before={\setlength{\parindent}{12.5mm}}
}

\setlist[enumerate]{
	leftmargin=\parindent,            % убираем стандартный отступ списка
	itemindent=0pt,     % основной текст начинается с красной строки
	labelwidth=10pt,           % выделяем место для номера
	labelsep=2.5mm,            % отступ между номером и текстом
	labelindent=\dimexpr\parindent-10mm-2.5mm\relax, % позиция номера (левее красной строки)
	align=left,
	parsep=0pt,
	itemsep=0pt,
	topsep=0pt,
	label=\arabic*),
	before={\setlength{\parindent}{12.5mm}}
}


% ---------------------------------------- TABLE  ----------------------------------------

\usepackage{xcolor}
\usepackage{tabularx}
\usepackage{booktabs}
\usepackage{multirow}
\usepackage{diagbox}
\usepackage{placeins}

\newcolumntype{O}{>{\centering\arraybackslash}p{0.08\textwidth}}
\newcolumntype{T}{>{\centering\arraybackslash}p{0.3\textwidth}}
\newcolumntype{L}{>{\centering\arraybackslash}p{0.45\textwidth}}
\newcolumntype{P}{>{\centering\arraybackslash}p{0.2\textwidth}}
\newcolumntype{R}{>{\centering\arraybackslash}p{0.22\textwidth}}
\newcolumntype{F}{>{\centering\arraybackslash}p{0.25\textwidth}}
\newcolumntype{S}{>{\centering\arraybackslash}p{0.15\textwidth}}
\newcolumntype{B}{>{\centering\arraybackslash}p{0.12\textwidth}}
\newcolumntype{U}{>{\centering\arraybackslash}p{0.16\textwidth}}
\newcolumntype{N}{>{\centering\arraybackslash}p{0.19\textwidth}}
\newcolumntype{H}{>{\centering\arraybackslash}p{0.13\textwidth}}
\newcolumntype{E}{>{\centering\arraybackslash}p{0.18\textwidth}}
\newcolumntype{G}{>{\centering\arraybackslash}p{0.6\textwidth}}
\newcolumntype{Q}{>{\centering\arraybackslash}p{0.1\textwidth}}
\newcolumntype{A}{>{\centering\arraybackslash}p{0.5\textwidth}}

% ---------------------------------------- FIGURE ----------------------------------------

\usepackage{graphicx}
\usepackage{float}
\usepackage{wrapfig}
\usepackage{tikzscale}
\usepackage{tikz}
\usetikzlibrary{calc}
\usetikzlibrary{shapes.misc}      % Для 'rounded rectangle'
\usetikzlibrary{shapes.geometric} % Для 'diamond'
\usetikzlibrary{positioning}      % Для синтаксиса 'below=of'
\usetikzlibrary{arrows.meta}      % Для более точного определения стрелок
\usepackage[notransparent]{svg}
\svgpath{{images/}}
\makeatletter
	\let\quote@name\unquote@name
\makeatother


\usepackage{pgfplots}
\pgfplotsset{compat=newest}

% ----------------------------------------- MATH -----------------------------------------

\usepackage{lscape}
\usepackage{afterpage}

\usepackage{amsmath}

\DeclareMathOperator*{\argmax}{arg\,max}
\DeclareMathOperator*{\argmin}{arg\,min}
% ----------------------------------------- LST -----------------------------------------
\usepackage{listings}
\usepackage{courier}
\usepackage{spverbatim}

\renewcommand{\lstlistingname}{Листинг}

\newcommand{\codefont}{\fontfamily{pcr}}
\newcommand{\keywordsfont}{\fontfamily{pcr}\bfseries}

%\usepackage{minted}
\lstset{
	basicstyle=\codefont\footnotesize,
	keywordstyle=\keywordsfont\color{black},
	numbers=left,
	numberstyle=\fontfamily{pcr}\tiny,
	showstringspaces=false,
	numbersep=10pt,
	tabsize=4,
	frame=tblr, 
	xleftmargin=25pt,
	framexleftmargin=18pt,
	framexrightmargin=-5pt,
	% framexbottommargin=10pt,
	% linewidth=0.95\pagewidth,
	float=H, % Makes listings float
	floatplacement=!ht, % Prevents listings from breaking across pages
	aboveskip=0pt, % Adds space above
	belowskip=0pt,
	breaklines=true,
	extendedchars=true,
	texcl=true,
	inputencoding=utf8,
	basicstyle=\cyrillicfonttt\footnotesize,
	columns=flexible,  % Гибкое распределение ширины столбцов
	keepspaces=true,   % Сохранять пробелы как есть
	basewidth={0.5em,0.5em}  % Настройка базовой ширины символов
}

\usepackage[ruled,linesnumbered,resetcount,algochapter]{algorithm2e}
\SetKwInput{KwData}{Исходные параметры}
\SetKwInput{KwResult}{Результат}
\SetKwInput{KwIn}{Входные данные}
\SetKwInput{KwOut}{Выходные данные}
\SetKwIF{If}{ElseIf}{Else}{если}{тогда}{иначе если}{иначе}{конец условия}
\SetKwFor{While}{до тех пор, пока}{выполнять}{конец цикла}
\SetKw{KwTo}{от}
\SetKw{KwRet}{возвратить}
\SetKw{Return}{возвратить}
\SetKwBlock{Begin}{начало блока}{конец блока}
\SetKwSwitch{Switch}{Case}{Other}{Проверить значение}{и выполнить}{вариант}{в противном случае}{конец варианта}{конец проверки значений}
\SetKwFor{For}{цикл}{выполнять}{конец$\;$цикла} % очевидный и невероятный костыль
\SetKwFor{ForEach}{для каждого}{выполнять}{конец цикла}
\SetKwRepeat{Repeat}{повторять}{до тех пор, пока}
\SetAlgorithmName{Листинг}{алгоритм}{Список алгоритмов}

% ----------------------------------------- BIBLIO ---------------------------------------
\usepackage{totcount}
\regtotcounter{table}    % для подсчета таблиц
\newcounter{citnum}
\regtotcounter{citnum}
\newtotcounter{appendix}  % создаем счетчик приложений


\usepackage[
	backend=biber,
	style=gost-numeric,
	language=auto,
	sorting=none,
	maxnames=999,
	doi=false,
	isbn=false,
	url=true,
	dashed=false
]{biblatex}

\makeatletter
\AtEveryBibitem{\stepcounter{citnum}}
\makeatother

% Базовые настройки
\usepackage{csquotes}
\addbibresource{ref-lib.bib}

% Настройка формата элементов
\DeclareFieldFormat{title}{%
	\ifentrytype{misc}{%
		#1 [Электронный ресурс]%
	}{%
		\ifentrytype{article}{%
			#1 [Электронный ресурс]%
		}{%
			#1%
		}%
	}%
}
\DeclareFieldFormat{url}{Режим доступа: #1}
\DeclareFieldFormat{urldate}{(дата обращения: \ifnum\thefield{urlday}<10 0\fi\thefield{urlday}.\ifnum\thefield{urlmonth}<10 0\fi\thefield{urlmonth}.\thefield{urlyear})}

% Настройка URL
\usepackage{url}
\urlstyle{same}

